\chapter{Theory and Fundamentals of MIMO and MCM}
The fundamental principles of \acrshort{mimo} and \acrshort{mcm} are presented here. The following pages firmly establishes some of the prerequisite learning required before the report can discuss the actual design of the system in the next chapter. This report begins by looking at the shortcomings of single channel systems and how they can be overcome with the help of \acrshort{mcm}. This report also highlights how some of the shortcomings of single channel systems can be used to our advantage in \acrshort{mcm}-\acrshort{mimo} systems. This report also spends some time looking into concepts such as \gls{rayleigh fading} and different precoding schemes to gain a thorough understanding on mobile communication systems.

\section{The Need for MIMO}

In a typical mobile communication system, the \acrlong{ue} (\acrshort{ue}) and \acrlong{bs} (\acrshort{bs}) are quite far apart. Typical distances are in the kilometer range. As a result, the signal undergoes attenuation as it travels and signal quality degrades to levels that make retrieval of information impossible. A typical example of the variation of received power with distance is given in the figure \ref{fig:path loss}.\\
Various statistical models have been developed to model path loss and fading. In our report, a simple \acrlong{los} path loss function (\acrshort{los}). This coupled with \gls{rayleigh fading} and \acrshort{awgn} noise forms the channel component of our report. The exact implementation details of each of these terms is discussed in Chapter 3.\\

Apart from this, there are also considerations for the service provider with regards to maximizing user capacity and spectral efficiency. Along with this, it becomes necessary to provide high data rates to subscribers for various applications. Meeting all these requirements in single channel systems leads to the design of extremely cost prohibitive modems and in some cases is almost impossible. Hence, one must come up with large scale and complex systems of transmitters and receivers to meet the demand. In this report, the implementation of various versions of \acrshort{mimo} upto $2 \times 2$ is shown. \acrshort{mimo} itself can be used in various ways namely multiplexing and diversity. In the following sections this report takes a closer look at these configurations.

\begin{figure}[!htbp]
\centering
\includegraphics[scale=1]{Chapter 2/Figures/Path Loss}
\caption[Signal Degradation due to Path Loss and Fading]{Variation of Received Signal Power with Distance of separation (d) between transmitter and receiver. Source \textcite{Ghosh2010}}
\label{fig:path loss}
\end{figure}

\section{Diversity}
Mitigation of fading and overcoming low \acrshort{snr} regimes to maintain a suitable \acrshort{qos} requires techniques such as diversity, wherein copies of the same data are sent from the transmitter to the receiver so that reliability of accurately decoding the transmitted symbols is increased.\\
Diversity can be achieved primarily in three ways, they are
\begin{enumerate}
\item \textbf{Frequency Diversity}: Where multiple copies of the same data are sent on different frequency channels
\item \textbf{Time Diversity}: Where multiple copies of the same data are sent at different instances of time
\item \textbf{Spatial Diversity}: Where multiple copies of the same data are sent along different antenna paths.
\end{enumerate}
Among the three possible methods, \gls{spatial diversity} is attractive to us because, the multiple reflections that a signal undergoes in a typical urban setup already provides us with the required diversity without the loss of bandwidth efficiency. Hence, when this report refers to diversity, unless otherwise mentioned, it is assumed to refer to \gls{spatial diversity}. A simplified illustration of multipath propagation in urban setting is shown in the figure \ref{fig:multipath propagation}\\
\begin{figure}[!htbp]
\centering
\includegraphics[scale=1]{Chapter 2/Figures/Multipath Propagation}
\caption[Multipath Propagation in a typical urban setting]{Multipath Propagation in a typical urban setting. Source \textcite{Ghosh2010}}
\label{fig:multipath propagation}
\end{figure}
However, in a single channel system, the multiple copies arriving at different time instances leads to interference of the signal. This interference may be constructive or destructive in nature as shown in the figure \ref{fig:constructive and destructive interference}. This can lead to difficulties in decoding as it would mean the requirement of expensive equalizers or reduction in the symbol rate. Neither option is feasible for us, and hence, it becomes apparent to us how having multiple transmit and receive antennas can easily overcome this issue. With the help of multiple antennas, the same situation which was causing \acrlong{isi} (\acrshort{isi}) becomes a boon to us by allowing multiple antenna paths between the transmitter and receiver allowing for easy implementation of \gls{spatial diversity}. This situation is shown in the figure \ref{fig:spatial diversity}\\

\section{Multiplexing}
Supposing the channel conditions are suitable and the \acrshort{snr} is sufficiently high, meaning one is in a high \acrshort{snr} regime, instead of sending multiple copies of the same data, one can send different data blocks on different antenna paths increasing the overall data rate per user and the user capacity of the system. This concept is known as \gls{spatial multiplexing} demonstrated in the figure \ref{fig:spatial multiplexing}.\\

Apart from \gls{spatial multiplexing} one can also implement time multiplexing and frequency multiplexing wherein different data symbols are sent in different time slots or frequency blocks respectively. Modern day \acrshort{4g} and \acrshort{5g} uses all three forms of multiplexing to increase data rates and capacity.\\

When one is given the various options for multiplexing, one can either
\begin{enumerate}
\item Assign multiple resource blocks (either in time, frequency or antenna paths) to a single user to significantly improve his data rate and \acrshort{qos}.
\item The alternative is to accommodate more users by assigning each one or more resource blocks to each and improving the capacity.
\end{enumerate}

This option is left to the service providers to implement resource allocation as per the market requirements. Hence, one sees the significant advantages \acrshort{mimo} has enabled for us by opening the doors to \acrshort{mimo} and \acrshort{mcm}.

\begin{figure}[!htbp]
\centering
\includegraphics[scale=1]{Chapter 2/Figures/Interference}
\caption[Constructive and Destructive Interference of Signals]{Constructive and Destructive Interference leading to large variation in received signal power. Source \textcite{Ghosh2010}}
\label{fig:constructive and destructive interference}
\end{figure}

\begin{figure}[!htbp]
\centering
\includegraphics[scale=0.6]{Chapter 2/Figures/Spatial Diversity}
\caption[Spatial Diversity]{Spatial Diversity. Source\textcite{Ghosh2010}}
\label{fig:spatial diversity}
\end{figure}

\begin{figure}[!htbp]
\centering
\includegraphics[scale=0.7]{Chapter 2/Figures/Spatial Multiplexing}
\caption{Spatial Multiplexing}
\label{fig:spatial multiplexing}
\end{figure}


\section{Intersymbol Interference, Frequency Selective Fading and the need for MCM}
In the previous section readers saw sufficient motivation to move in the direction of multiple antenna system. However, the issue of \acrshort{isi} needs to be tackled sufficiently to provide significantly high data  rates. Added to the menace of \acrshort{isi} the channel can also degrade the message in a frequency selective manner leading to added difficulties in information recovery at the receiver as shown in the figure \ref{fig:frequency selective fading}. Frequency selective fading occurs because the channel conditions are in constant flux and the message time period is not the same as the time period for which the channel conditions are relatively constant\\
An effective way to combat frequency selective fading is to breakup the entire bandwidth into smaller subchannels where the bandwidth of each subchannel is smaller than the \acrlong{bc} (\acrshort{bc}), thus ensuring that the message time period is smaller than the \acrlong{td} (\acrshort{td}). This approach to communication is called as \acrlong{mcm}(\acrshort{mcm}) technique. The implementation of \acrshort{mcm} is simple if one realizes that one can split the given bandwidth into different subchannels by simply introducing an \acrshort{ifft} block at the transmitter and to achieve the opposite effect introduce an \acrshort{fft} block at the receiver. With the help of this, one is able to significantly reduce the problems of frequency selective fading and \acrshort{isi}. The basic structure of a \acrshort{mcm} transmitter and receiver is given in the figures \ref{fig:mcm transmitter} and \ref{fig:mcm receiver}.

\begin{figure}[!htbp]
\centering
\includegraphics[scale=1]{Chapter 2/Figures/Frequency Selective Fading}
\caption[Frequency Selective Fading]{Frequency Selective Fading which occurs because the message is longer than the delay spread of the channel. Source \textcite{Ghosh2010}}
\label{fig:frequency selective fading}
\end{figure}

\begin{figure}[!htbp]
\centering
\includegraphics[scale=1]{Chapter 2/Figures/Flat Fading Subchannel}
\caption[Flat Fading Subchannel]{By breaking the large bandwidth into smaller subchannels, one can achieve an almost flat fading subchannel which is desirable. Source \textcite{Ghosh2010}}
\label{fig:flat fading subchannel}
\end{figure}

\begin{figure}[!htbp]
\centering
\includegraphics[scale=1]{Chapter 2/Figures/MCM Transmitter}
\caption[MCM Transmitter]{An MCM transmitter with an IFFT block to split the given bandwidth into smaller $L$ subchannels. Source \textcite{Ghosh2010}}
\label{fig:mcm transmitter}
\end{figure}

\begin{figure}[!htbp]
\centering
\includegraphics[scale=1]{Chapter 2/Figures/MCM Receiver}
\caption[MCM Receiver]{An MCM transmitter with an FFT block to reverse the effects of IFFT block at the transmitter. Source \textcite{Ghosh2010}}
\label{fig:mcm receiver}
\end{figure}




\section{Shortcomings of simple MCM and the need for OFDM}
Having shown the implementation of a simple \acrshort{mcm} system, this report now addresses some of the shortcomings of this. Primarily,
\begin{itemize}
\item It is impossible to realistically have sharply defined bandwidths, as there exists no way to define a pulse which is strictly rectangular in the frequency domain.
\item Expensive low pass filters will be necessary to maintain orthogonality of the subchannels.
\item Importantly, multiple \acrshort{rf} units are required at both ends for the system to work. This setup, as a result becomes unfeasible and thus, in the next section the reader shall look into the \acrshort{ofdm} scheme as an alternative to simple \acrshort{mcm}.
\end{itemize}

\section{OFDM}
\subsection{Concept of OFDM}
\acrlong{ofdm} is a multiplexing scheme where different data symbols are modulated to different frequencies. These frequencies are chosen such that they are all orthogonal. Hence a given instance, only one wave is at it's peak while the rest are at zero allowing us to read the data bits without any \acrlong{isi}. This situation is shown in the figure \ref{fig:ofdm orthogonal waves}.\\

Therefore, one clubs the different data bits into one block called an \acrshort{ofdm} symbol. To avoid \acrshort{isi} between the \acrshort{ofdm} symbols themselves, there is a small time delay introduced between the \acrshort{ofdm} symbols called as \acrlong{tg} which is abbreviated to \acrshort{tg}. It is important that this delay, is atleast as large as the \acrlong{td}.\\
One knows that the wireless channel behaves as a \acrlong{lti} system and hence, the channel coefficient and data bits are linearly convolved together whenever a message is passed through it. However, one knows that, circular convolution in the time domain yields simple multiplication in the frequency domain. This multiplication is desirable as it leads to simplified computation at the transmitter and receiver. A simple way to covert this linear convolution to circular convolution is to add redundant bits known as \gls{cyclic prefix}. This cyclic prefix is just copying the last $L$ bits of the \acrshort{ofdm} symbol and adding it to the beginning of the symbol. These $L$ bits are transmitted during the time \acrshort{tg} and hence will be lost due to interference between the \acrshort{ofdm} symbols.\\

\begin{figure}[!htbp]
\centering
\includegraphics[scale=1]{Chapter 2/Figures/OFDM Orthogonality}
\caption[Orthogonality in OFDM]{An OFDM symbol where different coloured waves correspond to different bits. Notice how when one wave peaks, all the other waves are at their null points. Source \textcite{Ghosh2010}}
\label{fig:ofdm orthogonal waves}
\end{figure}

\begin{figure}[!htbp]
\centering
\includegraphics[scale=1]{Chapter 2/Figures/OFDM Symbol Timing}
\caption[Guard Time between OFDM symbols]{A delay of \acrshort{tg} is introduced between the symbols to avoid interference between the \acrshort{ofdm} symbols. Notice that this does not do anything to combat \acrshort{isi} within the \acrshort{ofdm} symbol itself.}
\label{fig:ofdm symbol timing}
\end{figure}

\begin{figure}[!htbp]
\centering
\includegraphics[scale=1]{Chapter 2/Figures/Cyclic Prefix}
\caption[Cyclic Prefix]{The cyclic prefix in an \acrshort{ofdm} symbol. Source \textcite{Ghosh2010}}
\label{fig:ofdm cyclic prefix}
\end{figure}



\subsection{Advantages of OFDM}
Some of the advantages of \acrshort{ofdm} compared to traditional \acrshort{fdm} are as follows.
\begin{itemize}
\item There is no need for any guard bands between carriers leading to higher spectral efficiency.
\item Higher data rates can be achieved as symbol rate need not be lowered for the sake of \acrshort{isi}.
\item System is more robust to multipath effects.
\end{itemize}

\subsection{Disadvantages of OFDM}
\acrshort{ofdm} also comes with a few disadvantages chief among them is the issue of high \acrshort{papr}. Discussing the ways to mitigate this issue is outside the scope of this report and the reader is encouraged to refer to literature such as \textcite{Ghosh2010} to gain a better understanding.

\section{OFDM Transceiver System}
After having seen the motivation for the development of \acrshort{mcm}, \acrshort{mimo} and \acrshort{ofdm} schemes and also having seen a basic \acrshort{mcm} transceiver system, the different concepts are all combined to create an \acrshort{ofdm} transceiver which is capable of sending and receiving data bits packaged in \acrshort{ofdm} symbols.\\
In figure \ref{fig:ofdm transmitter} one can see how normal \acrshort{qam} modulated symbols are passed through an \acrshort{ifft} block to assign them to different frequency subchannels. Additional cyclic prefix is added before converting the parallel streams to a serial stream and transmitting it.\\
The \acrshort{ofdm} receiver in figure \ref{fig:ofdm receiver} on the other hand does the exact opposite process, where the received symbols are demodulated according to their respective frequencies and passed through an \acrshort{fft} block to undo the \acrshort{ifft} process. Then, it the demodulated symbols are passed through a \acrlong{mli} detector to get back the information bits.\\

\begin{figure}[!htbp]
\centering
\includegraphics[scale=1]{Chapter 2/Figures/OFDM Transmitter}
\caption[\acrshort{ofdm} Transmitter]{\acrshort{ofdm} Transmitter. Source \textcite{Ghosh2010}}
\label{fig:ofdm transmitter}
\end{figure}

\begin{figure}[!htbp]
\centering
\includegraphics[scale=0.6]{Chapter 2/Figures/OFDM Receiver}
\caption[\acrshort{ofdm} Receiver]{\acrshort{ofdm} Receiver. Source \textcite{Ghosh2010}}
\label{fig:ofdm receiver}
\end{figure}


\section{Alamouti Coding Scheme}
Alamouti coding scheme is a simple coding scheme designed for the purpose of achieving \gls{spatial diversity} in \acrshort{miso} systems. The advantage of this coding scheme is that the transmitter need not know the channel information before sending the data. This section describes the coding scheme.\\
Consider two transmitting antennas $T_1$ and $T_2$ and one receiving antenna $R$.\\
Let $h_1$ be the channel coefficient of the first antenna path and $h_2$ be the channel coefficient of the second antenna path.\\
Let $x_1$ and $x_2$ be transmitted by antennas $T_1$ and $T_2$ respectively at a given time instance, and ${-x_2}^*$, ${-x_1}^*$ be the data transmitted in the next time instance by the antennas respectivey.\\
It is known that the wireless channel behaves as an \acrshort{lti} system which performs convolution of the data bits and the channel coefficient. Also, let $w_1$ and $w_2$ be the noise vectors added at the two time instances respectively.\\
This situation can be represented mathematically as follows.
\begin{align}
y_1 &= 
\begin{bmatrix}
h_1&h_2
\end{bmatrix}
\times
\begin{bmatrix}
x_1\\
x_2
\end{bmatrix}
+ w_1\\
y_2 &=
\begin{bmatrix}
h_1&h_2
\end{bmatrix}
\times
\begin{bmatrix}
{-x_2}^*\\
x_1^*
\end{bmatrix}
+w_2
\end{align}
This can be further simplified as,\\
\begin{align}
y &= \begin{bmatrix}
y_1\\
y_2^*
\end{bmatrix}
= c_1x_1 + c_2x_2 + w
\end{align}
Where,
\begin{align}
c_1&=\begin{bmatrix}
h_1\\
h_2^*
\end{bmatrix}\\
c_2&=\begin{bmatrix}
h_2\\
-h_1^*
\end{bmatrix}
\end{align}
Here $c_1$ and $c_2$ can be shown as orthogonal in nature and so, this coding scheme is also known as \acrlong{ostbc} scheme.\\
At the receiver, once the matrix $y$ is obtained, $x_1$ and $x_2$ can be recovered as follows,
\begin{align}
\frac{c_1^H}{||c_1||} \cdot y &= ||c_1||x_1 + \overline{w_1}\\
\frac{c_2^H}{||c_2||} \cdot y &= ||c_2||x_2 + \overline{w_2}
\end{align}
Here, $c_1^H$ and $c_2^H$ are the result obtained after performing the Hermitian operator on the $c$ matrices. It is noticed that $x_1$ and $x_2$ are scaled by a factor and mixed with \acrshort{awgn} noise. With a suitable decision criteria, both $x_1$ and $x_2$ can be decoded correctly in two time slots. This shows how Alamouti coding scheme is effective when used in the \gls{spatial diversity} mode of a $2 \times 1$ \acrshort{miso} system. However, in higher order schemes, Alamouti coding loses it's efficiency and is not feasible.

\section{Inverse Channel Estimation}
The received symbol vector $y$ is related to the transmitted vector $x$ and channel coefficient matrix $h$ as\\
\begin{align}
\begin{bmatrix}
y
\end{bmatrix}
&=
\begin{bmatrix}
x
\end{bmatrix}
\times
\begin{bmatrix}
h
\end{bmatrix}
+
\begin{bmatrix}
n
\end{bmatrix}
\end{align}
where $\begin{bmatrix} n \end{bmatrix}$ is the \acrshort{awgn} noise vector.\\
From this, one can extract the sent symbols by multiplying $\begin{bmatrix} y \end{bmatrix}$ with the inverse of the channel coefficient matrix which yields the following expression.

\begin{align}
\begin{bmatrix}
y
\end{bmatrix}
\times
\begin{bmatrix}
h
\end{bmatrix}
^{-1}
&=
\begin{bmatrix}
x
\end{bmatrix}
+
\begin{bmatrix}
w_1
\end{bmatrix}
\end{align}
where $\begin{bmatrix} w_1 \end{bmatrix}$ is the result obtained after multiplying $\begin{bmatrix} n \end{bmatrix}$ with $\begin{bmatrix}h \end{bmatrix}^{-1}$.\\

Further processing is necessary to remove the noise vector $\begin{bmatrix} w_1 \end{bmatrix}$. However, one of the issues that us faced is that $\begin{bmatrix} w_1 \end{bmatrix}$ is no longer \acrshort{awgn} but becomes colored noise. Therefore, to overcome this hindrance,  precoding of the transmitted symbols vector $\begin{bmatrix} x \end{bmatrix}$ is done by multiplying it with $\begin{bmatrix} h \end{bmatrix}^{-1}$. The received vector $\begin{bmatrix} y \end{bmatrix}$ then becomes

\begin{align}
\begin{bmatrix} y \end{bmatrix} &=
\begin{bmatrix} x \end{bmatrix}
\times
\begin{bmatrix} h \end{bmatrix}^{-1}
+
\begin{bmatrix} n \end{bmatrix}
\end{align}

Finally, when one tries and extracts the transmitted symbols at the receiver by multiplying with $\begin{bmatrix} h \end{bmatrix}$ one gets

\begin{align*}
\begin{bmatrix} y \end{bmatrix} \times
\begin{bmatrix} h \end{bmatrix} &=
\times
\begin{bmatrix} x \end{bmatrix}
+
\begin{bmatrix} w_2 \end{bmatrix}
\end{align*}
where $\begin{bmatrix} w_2 \end{bmatrix}$ is the result obtained after multiplying $\begin{bmatrix} n \end{bmatrix}$ with $\begin{bmatrix} h \end{bmatrix}$.\\

Now, $\begin{bmatrix} w_2 \end{bmatrix}$ remains to be Gaussian and hence further noise processing becomes simpler with normal demodulators and \acrshort{mli} estimators.




\section{Singular Value Decomposition}
In the previous section readers saw the advantage of using Inverse Channel Estimation precoding. However, in massive \acrshort{mimo} systems where the number of antenna paths are plenty and the order of the channel coefficient matrix is large, inversion of matrices becomes a computationally intensive task. Since it is required to have high speed \gls{modems} which do not take more than a few microseconds to make the necessary computations, one must look to faster ways of inverting the channel coefficient matrix.\\
In this effort this report uses the singular value decomposition technique where one decomposes the channel coefficient matrix $\begin{bmatrix} h \end{bmatrix}$ into three orthogonal matrices $\begin{bmatrix} U \end{bmatrix}$ , $\begin{bmatrix} \Sigma \end{bmatrix}$ and, $\begin{bmatrix} V \end{bmatrix}$.

At the transmitter one multiplies the transmitted symbol vector $\begin{bmatrix} x \end{bmatrix}$ with $\begin{bmatrix} V \end{bmatrix}$ and similarly at the receiver one multiplies the received vector $\begin{bmatrix} y \end{bmatrix}$ with $\begin{bmatrix} U \end{bmatrix}$ to achieve the same effect as inversion.\\
Singular value decomposition is faster than regular inversion for large matrices and hence proves faster in massive \acrshort{mimo} systems.

\section*{Summary}
This chapter has elaborated on the motivations behind \acrshort{mcm} and \acrshort{mimo}. The authors have also clearly elaborated on the key technologies that enable them. The next chapter discusses the implementation details of the \acrshort{mcm}-\acrshort{mimo} system and explains the different algorithms used. Finally, in Chapter 4 the results obtained after simulations are discussed and conclusions are drawn as to the overall system performance.  
