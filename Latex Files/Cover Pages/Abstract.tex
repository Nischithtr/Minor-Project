\addcontentsline{toc}{chapter}{Abstract}\vspace{-1cm}
%Border
\begin{tikzpicture}[remember picture, overlay]
  \draw[line width = 4pt] ($(current page.north west) + (0.75in,-0.75in)$) rectangle ($(current page.south east) + (-0.75in,0.75in)$);
\end{tikzpicture}



Mobile communications and telephony has become the norm of modern day communication with digital wireless communication systems almost fully having replaced analog communication systems. The challenges we predominantly face in wireless systems are that of improving channel capacity in a manner that utilizes the available spectrum efficiently and manages to achieve a high data rate comparable to that of wired systems so that users can experience the various applications and services requiring high data rates. In this report, we look at how we can improve on the capacity and data rates of wireless systems with the help of \acrlong{mcm}, which is abbreviated as \acrshort{mcm} and \acrlong{mimo}, which is abbreviated as \acrshort{mimo} technologies.\\



In this report, we develop the methodology to use current \acrshort{4g} and future \acrshort{5g} systems in both diversity and multiplexing modes to improve on both data rates and capacities. We use methods such as \acrlong{svd}, which is abbreviated \acrshort{svd} and Alamouti coding to achieve these functionalities. We have also developed our own tone loading algorithms to help with the same.\\ 



We show the simulations of our work with the help of \gls{matlab} and all the code files are attached with this report for examination. In all simulations, we have used test-cases which try and match the real world as closely as possible with a good degree of statistical accuracy. Through this report we see that it is possible to easily upgrade existing systems to work in both diversity and multiplexing modes to achieve good improvements in capacity and data rates.




\pagebreak