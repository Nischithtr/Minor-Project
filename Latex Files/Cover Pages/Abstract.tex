\addcontentsline{toc}{chapter}{Abstract}\vspace{-1cm}
%Border
\begin{tikzpicture}[remember picture, overlay]
  \draw[line width = 4pt] ($(current page.north west) + (0.75in,-0.75in)$) rectangle ($(current page.south east) + (-0.75in,0.75in)$);
\end{tikzpicture}



Modern day mobile communication systems rely heavily on \acrlong{mcm} (\acrshort{mcm}) and \acrlong{mimo}(\acrshort{mimo}) technologies. The challenges predominantly faced are of maximizing data rates and capacity. We will examine in this report how to overcome these challenges through \acrshort{mcm} and \acrshort{mimo}. \acrshort{mimo} however is too broad a term as it consists of various configurations and setups. In this report we examine $2 \times 1$ \acrshort{miso} in diversity mode and $1 \times 2$ \acrshort{simo} and $2 \times 2$ \acrshort{mimo} in both diversity and multiplexing modes.\\




In this report, we develop the methodology to use current \acrshort{4g} and future \acrshort{5g} systems in both diversity and multiplexing modes to improve on both data rates and capacities. Theoretically, it is possible to use \acrlong{siso}(\acrshort{siso}) systems to achieve the same capacities and data rates. However, the complexities involved in designing \gls{modems} that are capable of achieving these rates are too cost prohibitive and thus it becomes necessary for us to address this problem through \acrshort{mcm} and \acrshort{mimo}.\\

We design a system which can be operated in two modes, namely diversity and multiplexing. Diversity is suitable for low \acrshort{snr} regimes where \acrlong{ber}(\acrshort{ber}) could be high, so multiple copies of the same data needs to be sent to maintain a feasible \acrlong{qos}(\acrshort{qos}). Multiplexing is the scheme of choice in high \acrshort{snr} regimes where we have the option of maximizing data rates and capacity by transmitting multiple copies over good channels that are already providing low \acrshort{ber}.


We show the simulations of our work with the help of \gls{matlab}. In our results, we show that through an optimal tone loading algorithm we are able to achieve a \acrshort{ber} close to zero. As expected we observe that data rates in multiplexing modes are higher (almost doubled) than that of in diversity mode. We also show that by  using sufficient precoding methods like Alamouti, Inverse Channel Decomposition and \acrshort{svd} we improve the speed with which our modem is able to transmit and receive data. Through these results, it becomes clear to us that our system could be easily used to improve upon existing \acrshort{4g} and \acrshort{5g} systems to improvise on capacity and data rates.


\pagebreak