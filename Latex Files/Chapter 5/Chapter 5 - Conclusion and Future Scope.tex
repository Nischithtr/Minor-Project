\chapter{Conclusion and Future Scope}
This report has examined how one can improve upon the capacity and data rates of existing \acrshort{4g} and future \acrshort{5g} systems with the help of \acrshort{mcm}-\acrshort{mimo} systems.\\

In this report a system is designed which can be operated in two modes as stated, namely in diversity and multiplexing modes. It is shown how diversity is suitable for low \acrshort{snr} regimes, so multiple copies of the same data needs to be sent to maintain a feasible \acrlong{qos}(\acrshort{qos}). This report also shows how multiplexing is suitable in high \acrshort{snr} regimes where one has the option of maximizing data rates and capacity by transmitting differing data over good channels that are already providing good \acrshort{ber}.\\


The ultimate goal of this report was to design a $2 \times 2$ \acrshort{mimo} system and in this regard, at first a \acrshort{siso} system was designed to lay the foundation for multi carrier communication systems. Further, a $1 \times 2$ \acrshort{simo} and $2 \times 1$ \acrshort{miso} system was also designed to operate in diversity mode. Finally, a $2 \times 2$ \acrshort{mimo} system was designed to work in both diversity and multiplexing modes to improve the data rates and capacities. The design of this system employed unique precoding schemes like Alamouti coding, Inverse Channel Estimation precoding and \acrshort{svd} precoding. This report also demonstrated the use of \gls{rayleigh fading} and Friis' path loss formula for real world simulation purposes.

\section*{Summary}
For readers, it is suggested suggest further improvements can be done by improving upon the channel modeling by introducing more real world phenomenon like shadowing and use more complex path loss functions like Ricean distribution.\\
Another scope for improvisation is in the scope of the system. Here, the report is limited to $2 \times 2$ \acrshort{mimo} systems. Readers are encouraged to implement large systems leading to massive \acrshort{mimo} systems.\\


